\section{Introduction}

Human-Robot Collaboration (HRC) is a research topic becoming increasingly important in modern industry, driven by the need to enhance productivity, efficiency, and safety in work environments. The combination of human skills and robotic capabilities provides significant potential to improve the execution of complex and repetitive tasks. However, effective synchronization of actions and seamless communication between partners are open challenges that need to be further addressed.

In collaborative scenarios, assistive robots are designed to work alongside humans in assembly processes or maintenance operations, providing timely support to enhance the overall efficiency of the task. Robots can assist the human worker by delivering a component, tool, or part, by holding a part while the operator works on it, or by performing autonomously a specific sub-task. In any case, the ability of an assistive robot to anticipate the upcoming needs of a human operator plays a pivotal role in supporting efficient teamwork. By anticipating human intentions, actions, and needs, robots can proactively assist or complement human tasks, providing timely support and improving overall efficiency.

The research fellowship focused on advancing collaborative robotics through three main areas:
\begin{enumerate}
    \item \textbf{Human intention anticipation}: Optimization of a Convolutional Neural Network (CNN) classifier\cite{Amaral2023} for grasped objects recognition based on human hand posture data (keypoints) extracted from MediaPipe software. Integration of this object recognition module in a real-time anticipation system allowing the robot to predict the operator's needs.

    \item \textbf{Deep reinforcement learning for robot control}: Development of a Deep Reinforcement Learning (DRL) system to train a robotic arm (UR10e) in a simulated environment to move the robot's gripper so that it aligns itself with a target object represented by their 6D pose (position and orientation).

    \item \textbf{Exploration of control techniques beyond MoveIt}: Research of alternatives to MoveIt for robot control, with a focus on the implementation of joint-velocity-based control, and cartesian-velocity-based control modes.
\end{enumerate}

The remainder of the report is organised as follows. \Cref{section:system_architecture} describes the hardware, software, and specific details of the setup used for the work carried out. \Cref{section:human_intention_anticipation} presents the human intention anticipation system, detailing the object recognition module and the real-time anticipation system. \Cref{section:deep_reinforcement_learning} describes the deep reinforcement learning system for robot control, including the simulation environment, the DRL model, and the training process. \Cref{section:robot_control_techniques} presents the exploration of control techniques beyond MoveIt, detailing the joint-velocity-based control and cartesian-velocity-based control modes. Finally, \Cref{section:conclusions} summarises the main contributions of the research fellowship and outlines future work directions.
